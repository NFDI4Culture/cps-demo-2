\usepackage{hyphenat}
\usepackage{graphicx}
\usepackage{wallpaper} % for the background image on title page
\usepackage{geometry} % for margins
\usepackage{libertine}
\usepackage[most]{tcolorbox}
\usepackage{setspace}
\usepackage{parskip}
\usepackage{contour}
\usepackage{imakeidx}
\usepackage{ragged2e} % text alignment
\usepackage{titlesec}
\usepackage{titling}   % Für den Dokumenttitel


\setcounter{tocdepth}{0}
\makeindex[intoc=true]
\onehalfspacing % line spacing
\setlength{\parskip}{1.5em}
\setlength\parindent{0pt}


\setmainfont{Lato}[
    Path=./fonts/Lato/,
    Extension = .ttf,
    UprightFont=*-Regular,
    BoldFont=*-Bold,
    ItalicFont=*-Italic,
    BoldItalicFont=*-BoldItalic
]

% Überschriften-Schrift aus Datei laden (Pfad anpassen falls nötig)
\newfontfamily\headingfont{Kanit}[
    Path=./fonts/Kanit/,
    Extension = .ttf,
    UprightFont = *-Regular,
    BoldFont = *-SemiBold, 
    ItalicFont = *-Italic, 
    BoldItalicFont = *-SemiBoldItalic
]  

% Titel-Stil ändern
\pretitle{\begin{center}\headingfont\Huge\bfseries}
    \posttitle{\end{center}}
    \preauthor{\begin{center}\headingfont\Large}
    \postauthor{\end{center}}
    \predate{\begin{center}\headingfont\large}
    \postdate{\end{center}}
    
    % Kapitel-Formatierung ändern
    \titleformat{\chapter}[display]
      {\headingfont\Huge\bfseries}  % Stil der Kapitelüberschrift
      {\chaptername\ \thechapter}{10pt}{\headingfont}
    
    % Part-Formatierung ändern
    \titleformat{\part}
      {\headingfont\Huge\bfseries}  % Stil für Parts
      {\thepart}{20pt}{}
    

% Überschriften-Formatierung
\titleformat{\section}{\headingfont\LARGE\bfseries}{\thesection}{1em}{}
\titleformat{\subsection}{\headingfont\Large\bfseries}{\thesubsection}{1em}{}
\titleformat{\subsubsection}{\headingfont\normalsize\bfseries}{\thesubsubsection}{1em}{}

